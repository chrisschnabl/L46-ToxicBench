\section{Background}
\begin{table}[ht] \centering \begin{tabular}{@{}p{5.41cm}p{0.55cm}p{2.69cm}p{0.6cm}p{0.5cm}p{2.25cm}@{}}
\toprule
\textbf{Name} & \textbf{\#} & \textbf{Context} & \textbf{Rate} & \textbf{Label} & \textbf{Category} \\
\midrule
ToxicChat~\cite{ToxicChat} & 10k & Chatbot arena & 7\% & 0/1 & AI-Human \\
\midrule
RTPrompts~\cite{rtp} & 100k & Web text prompts & 14\% & 0$-$1 & Human-Human \\
\midrule
JigsawToxicity~\cite{jigsaw} & 224k & Social media & 9.6\% & 0/1 & Human-Human \\
\midrule
CivilComments~\cite{civilcomments} & 1.8M & Social media & ~5\% & 0$-$1 & Human-Human \\
\midrule
ToxiGen~\cite{hartvigsen2022toxigenlargescalemachinegenerateddataset} & 274k & Minority groups & 50\% & 0/1 & Generated \\
\midrule
ConvAbuse~\cite{cercas-curry-etal-2021-convabuse} & 13k & AI interactions & ~20\% & 0/1 & AI-Human \\
\midrule
HarmfulQA~\cite{bhardwaj2023redteaming} & 1k & Harmful questions & 100\% & 0/1 & Human-Human \\
\midrule
FFT~\cite{cui2024fftharmlessnessevaluationanalysis} & 2k & Evaluation & 100\% & 0/1 & AI-Human   \\
\midrule
SaladBench~\cite{li2024salad} & 40k & Safety-related tasks & Varies & 0/1 & Human-Human \\
\midrule
ImplicitToxicity~\cite{wen2023unveilingimplicittoxicitylarge} & 4k & RL-adverserial & 100\% & 0/1 & Generated \\
\bottomrule
\end{tabular} \vspace{0.25cm} \caption{Overview of various toxicity detection datasets with attributes such as number of samples, context, toxicity rate, label type, and category} \label{tab:toxicity_datasets} \end{table}
The rise of large language models has transformed online communication and created new challenges for toxicity detection~\cite{rtp, ToxicChat}. Existing methods, developed for social media content moderation, now face an expanded scope of toxic behaviors across human-AI interactions, machine-generated content, and social media platforms. Early datasets like JigsawToxicity captured 223,549 social media comments with a 9.6\% toxicity rate, focusing on obvious forms of harmful content~\cite{jigsaw}. As online communication evolved, CivilComments expanded this work by collecting 1.8 million comments with a lower 5\% toxicity rate, revealing more subtle forms of harmful content~\cite{civilcomments}. The field can be categorized into three types of datasets: machine-generated datasets created using AI models to evaluate implicit or adversarial toxicity~\cite{hartvigsen2022toxigenlargescalemachinegenerateddataset, wen2023unveilingimplicittoxicitylarge}, human-AI datasets capturing interactions with conversational systems to detect abuse or toxicity~\cite{ToxicChat, cercas-curry-etal-2021-convabuse}, and human-human datasets involving human-written content to identify explicit and nuanced toxic language~\cite{civilcomments, jigsaw}.
The emergence of language models introduced additional complexity. RealToxicityPrompts demonstrated this by collecting 100,000 web text prompts, finding a 14\% toxicity rate at a 70\% threshold. More importantly, they discovered that language models could generate toxic content even from benign prompts, highlighting the need for more sophisticated detection methods~\cite{rtp}. ToxicChat revealed a critical gap by analyzing 10,166 real user prompts to an open-source Vicuna chatbot, finding only 7.22\% contained toxic content, significantly lower than social media datasets. However, existing models failed to detect this toxicity effectively due to domain mismatch between social media training data and actual user-AI conversations~\cite{ToxicChat}. ConvAbuse reinforced these findings with 12,800 user-AI interactions, showing 20\% contained abusive content following patterns distinct from traditional social media toxicity, demonstrating that user-AI abuse requires specialized detection approaches~\cite{cercas-curry-etal-2021-convabuse}. ToxiGen addressed machine-generated toxicity by using GPT-3 to generate 274,186 statements about minority groups, maintaining a balanced 50\% toxicity rate. Their human evaluators confirmed the quality, labeling 94.5\% of toxic examples as genuine hate speech~\cite{hartvigsen2022toxigenlargescalemachinegenerateddataset}. ImplicitToxicity took a different approach, using reinforcement learning to generate toxic content that evades detection. They optimized a reward model to produce subtle toxicity hidden within seemingly normal language, creating examples that standard classifiers consistently miss~\cite{wen2023unveilingimplicittoxicitylarge}.

Current approaches primarily rely on fine-tuning. ToxicChat's authors fine-tuned RoBERTa-base on different datasets, finding that models trained on user-AI interactions significantly outperformed those trained on social media data for chatbot scenarios~\cite{ToxicChat}. ImplicitToxicity demonstrated that fine-tuning existing classifiers on their generated examples improved detection of subtle toxic content~\cite{wen2023unveilingimplicittoxicitylarge}.

Research is dominated by the following open problems. Domain adaptation remains hard as ToxicChat showed that social media-trained models fail on user-AI conversations, with significant drops in precision and recall~\cite{ToxicChat}. Evasion techniques are another challenge, where users develop "jailbreaking" prompts which results in an arm race between detection systems and adversarial users~\cite{hartvigsen2022toxigenlargescalemachinegenerateddataset}. Implicit content is hard to detect for standard classifiers as shown by by ImplicitToxicity's attacks against detection systems~\cite{wen2023unveilingimplicittoxicitylarge}. Annotation consistency is hard, as different datasets use varying annotation approaches~\cite{hartvigsen2022toxigenlargescalemachinegenerateddataset}.